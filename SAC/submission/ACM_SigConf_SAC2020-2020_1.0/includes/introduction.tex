\section{Introduction}\label{sec:introduction}

Representing words as vectors, i.e word embeddings (WE) is a fundamental aspect of Natural Language Processing (NLP). There are two ways to create such WE, either arbitrarily or with the purpose of capturing the semantic meaning of the words, i.e. vector representations of words that are syntactically or semantically similar will be close to each other in the vector space. By capturing semantic or syntactic meaning WE have shown to facilitate a lot of subsequent NLP tasks, such as entity recognition, machine translation or sentence classification. 
%TODO CITE WORK 
The first attempt to create WE with neural networks was made by Bengio et al. \citep{bengio} but more recently Mikolov et al. \citep{mikolov} introduced a software package called w2vec, which uses a simpler network that produces state of the art results. One of the proposed algorithms in this software package is the Skip-Gram Model (SGM). The SGM is an algorithm, that trains a neural network, on the task of predicting the neighboring words in a sentence. The weights of this network are then used as WE. Mikolov et al. \citep{mikolov2} then introduced an alteration to the SGM called negative sampling. The goal of the Skip-Gram Model with negative sampling (SGNS) is to distinguish for a given word $w$, context words $c_i$ (words that appear close to $w$ in a sentence) from words drawn randomly (i.e negative samples) $k_i$. This is achieved by maximizing the dot product of $w$ and $c_i$ and minimizing the dot product of $w$ and $k_i$.\\
The SGM, espacially the SGNS, gained a lot of attention, as it achieved very good results for a very simplistic model. As a consequence, a lot of effort went into optimizing it. Most of this effort was trying to improve the throughput of the model, i.e the number of words that are processed per second. The SGM uses Stochastic Gradient Descent as an optimization algorithm and is therefore inherently sequential. To remedy this problem Mikolov et al. used Hogwild! \citep{hogwild}, proposed by Recht et al., where different threads can access a shared model and update it. As this is not an optimal solution Yi et al. \citep{intel} tried to optimize it, by using a mini-batch like approach and converting vector to vector multiplications into matrix to matrix multiplications. This yielded two consequences: First the model is updated less frequently leading to less overwriting and offering the possibility to parallelize more. Secondly, it transformed level-1 BLAS operation into level-3 BLAS operations, and the algorithm could therefore effectively use computational resources. Another attempt at optimizing the throughput was made by Seulki and Youngmin \citep{gpu}. Their goal was to parallelize the algorithm on GPU's. They, therefore, chose to parallelize the update of the dimensions of each word representation. Both of these approaches and most of the literature are focused on improving the throughput of the model, but not the convergence time.  Therefore one could ask if the convergence time of the SGM can be optimized while at the same time maintaining its accuracy.\\
We propose a slightly altered version of the original SNGS, in which we compute the loss for a very large amount of training samples, i.e 2000. The loss of such a batch was computed by taking the sum over the loss all individual training samples in the specific batch. We combined this batched approach with advanced optimizers and input shuffling  This work focused on demonstrating that a batch approach, combined with advanced optimizers and input shuflling, does not hinder performance nor convergence time. If combined with an optimized throughput these results could lead to an overall decrease in runtime.\\
 The rest of the paper is structured as follows: section \ref{sec:background}  describes the SGM and the SGNS followed by its optimzations in Section \ref{sec:related_work}.  The reader is introduced to our Implementation in Section \ref{sec:contribution}. Furthermore  we will describe the used datasets, the measure applied to compare the quality of word embeddings and input shuffling in Section \ref{sec:evaluation}. Results are presented in  Section \ref{sec:results}. The last part will focus on the discussion of our results, and possible future work in Section \ref{sec:discussion} followed by a conclusion in Section \ref{sec:conclusion}.