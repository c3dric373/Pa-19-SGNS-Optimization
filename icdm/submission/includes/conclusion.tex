
\section{Conclusion}\label{chap:conclusion}

This work provides an overview of the Skip Gram Model with negative Sampling (SGNS) and the numerous successful attempts of optimizing the throughput of the model. As this is the case, no effort went into optimizng the convergence time of the SGNS, therefore this work focused on this point. We decided to use advanced optimizers and input shuffling as optimizing techniques. After giving a short overview over Gradient Descent algorithms this work proposes a slighlty altered version of the SGNS, where the idea is to compute the loss over the sum of a high number of training samples, i.e 2000,  instead of computing it for each individually.  We did this as it allowed us to compute more models and analyze the convergence time faster. We used the text8 dataset and used  word similarity as a quality measure for the word embeddings (WE). We used the State of the art implementation Gensim to compare our self. We did achieve a better convergence time than gensim with Adam as an optimizer and the use of input shuffling. Gensim convereged in 4 epochs to a word similarity of 0.66 and our model only took 2 epochs to achieve the same quality. Those results still need to be confirmed with more datasets. Finally, if this work would be combined with an optimized throughput it  could improve the state of the art run time of the SGNS.