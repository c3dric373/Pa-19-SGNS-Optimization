
\section{Introduction}\label{chap:introduction}

Representing words as vectors, i.e word embeddings (WE) is a fundamental aspect of Natural Language Processing (NLP). There are two ways to create such WE, either arbitrarily or with the purpose of capturing the semantics of the words, i.e. vector representations of words that are syntactically or semantically similar will be near to each other. By capturing semantic or syntactic meaning WE have shown to facilitate a lot of subsequent NLP tasks, such as entity recognition, machine translation or sentence classification.  The first attempt to create WE with neural networks was made by Bengio et al. \cite{bengio} but more recently Mikolov et al. \cite{mikolov} introduced a software package called w2vec that uses a simpler network and with this approach Mikolov et al. produced state of the art results. One of the proposed algorithms in this software package is the Skip-Gram Model (SGM). The SGM is an algorithm, that trains a neural network, on the task of predicting the neighboring words in a sentence. The weights of this network are then used as WE. 

The SGM gained a lot of attention, as it achieved very good results for a very simplistic model. As a consequence, a lot of effort went into optimizing it. Most of this effort was trying to improve the throughput of the model, i.e the number of words that are processed per second. The SGM uses Stochastic Gradient Descent as an optimization algorithm and is therefore inherently sequential. To remedy this problem Mikolov et al. used Hogwild \cite{hogwild}, where different threads can access a shared model and update it. As this is not an optimal solution Yi et al. \cite{intel} tried to optimize it, by using a mini-batch like approach and converting vector to vector multiplications into matrix to matrix multiplications. This yielded two consequences: First the model is updated less frequently leading to less overwriting and offering the possibility to parallelize more. Secondly, it transformed level-1 BLAS operation into level-3 BLAS operations, and the algorithm could therefore effectively use computational resources. Another attempt at optimizing the throughput was made by Seulki and Youngmin \cite{gpu}. Their goal was to parallelize the algorithm on GPU's. They, therefore, chose to parallelize the update of the dimensions of each word representation. Both of these approaches and most of the literature are focused on improving the throughput of the model, but not the convergence time.  Therefore one could ask if the convergence time of the SGM can be optimized while at the same time maintaining its accuracy. This work proposes an approach that uses advanced optimizers and input shuffling to optimize the convergence time of the SGM. In combination both of these techniques allowed us to decrease the convergence time of the SGM. If combined with an optimized throughput these result could lead to an overall decrease in runtime.

This work is structured as follows: subsection \ref{chap:background}  describes the SGM and its optimizations. Furthermore, we will give an extended explanation of the gradient descent optimizers used in this work. The reader is introduced to our Implementation in subsection \ref{chap:implementation}. Results are presented in  subsection \ref{chap:results} , where we will also describe used datasets, the measure applied to compare the quality of word embeddings, and finally the empirical results. The last part will focus on the discussion of our results, and possible future work in subsection \ref{chap:discussion} followed by a conclusion in subsection \ref{chap:conclusion}.